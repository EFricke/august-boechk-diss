%!TEX root = ../document-diss.tex

\chapter[Infos]{Sich wiederholende Informationen mit leichter Änderungen erstelle ich gerne mit LuaTeX}
\section{Einstiegsbeispiel}


Bekannte Archäologen sind diese:



\DTLdisplaydb{archaeologen}%Zeige DB als Tabelle

\DTLsort{Todesjahr=descending}{archaeologen}%bestimmte Reihenfolge

\DTLdisplaydb{archaeologen}

\begin{itemize}
\DTLforeach{archaeologen}
{\nachname=Nachname,%
	\name=Name,%
	\geburtsjahr=Geburtsjahr,%
	\todesjahr=Todesjahr%
}%
{\item[\theDTLrowi .] \name\xspace \nachname;%für Nummerierung, i = 1. Spalte
geboren im Jahr \geburtsjahr\ifdefempty\todesjahr{.}{;
gestorben \todesjahr.}%
}
\end{itemize}

%---
% Bitte hier eine Liste erstellen:
% Dann nach dem Todesjahr sotieren.
% Die Daten findest du in der csv-Datei archaeologen.csv
% ---

