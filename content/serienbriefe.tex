%!TEX root = ../document-diss.tex

\chapter{Serienbriefe, die zweite}

\section{Studierende an der HU}

An der Humboldt-Universität zu Berlin gibt es viele Studierende:
Klassische Archäologie (54 im Bachelor, 23 im MA,18 promovieren), 
Alte Geschichte (98 BA, 58 MA, 28 schreiben an der Doktorarbeit),
Klassische Philologie hat 78 Bachelors, 49 im Masterstudium und 23 stecken in der Promotion,
anders in der Ägyptologie: 49 im Grundstudium, 57 im Aufbaustudium und 14 herangehende Doktoren.

\section{Basic}

\DTLdisplaydb{fachstudenten}

\DTLforeach{fachstudenten}
{\fach=Fach,%
 \ba=BA,%
 \ma=MA,%
 \phd=Phd%
}%
{\begin{description}
  \item[\large\fach]~%Fake-Leerzeichen
 \begin{labeling}{Promotion}
 	\item[Bachelor]\ba
 	\item[Master]\ma
 	\item[Promotion]\phd
 \end{labeling}
\end{description}%
}

\begin{figure}[h]
\DTLbarchart%
  {variable=\ba,%
  barlabel=\fach,
  upperbarlabel=\ba}
  {fachstudenten}%Datebankname
  {\ba=BA,%
  \fach=Fach}
\caption{Anzahl der Bachelor-Studierenden}
\end{figure}

\begin{figure}[h]
	\DTLbarchart%
	{variable=\ma,%
		barlabel=\fach,
		upperbarlabel=\ma}
	{fachstudenten}%Datebankname
	{\ma=MA,%
		\fach=Fach}
	\caption{Anzahl der Master-Studierenden}
\end{figure}

\begin{figure}[h]
	\DTLbarchart%
	{variable=\phd,%
		barlabel=\fach,
		upperbarlabel=\phd}
	{fachstudenten}%Datebankname
	{\phd=Phd,%
		\fach=Fach}
	\caption{Anzahl der Promotion-Studierenden}
\end{figure}

\begin{figure}[h]
\DTLsetpiesegmentcolor{1}{teal}
\DTLsetpiesegmentcolor{2}{gray}
\DTLsetpiesegmentcolor{3}{purple}
\DTLsetpiesegmentcolor{4}{blue}
	\DTLpiechart%
{variable=\ba,%
outerlabel=\fach,%
innerlabel={\DTLpiepercent\%},%
cutaway={2},%für ein Segment herausstellen, Zahl nach Sortierung in der Datenbank
rotateinner%auch möglich: rotateouter -> äußere Bschriftung strahlenförmig
}
{fachstudenten}
{\ba=BA,\fach=Fach}
\end{figure}


\begin{figure}
\DTLsetbarcolor{2}{gray}
\DTLbarwidth=.5cm
	\DTLmultibarchart%
{variables={\ba,\ma,\phd},%
 barlabel=\fach,%
 %vericalbars=false,%horizontale Balken
 uppermultibarlabels={BA (\ba),MA (\ma),PhD (\phd)}}
{fachstudenten}
{\ba=BA,\ma=MA,\phd=Phd,\fach=Fach}
\end{figure}

%----
% Diese Informationen möchte ich gerne auf unterschiedliche Arten zeigen:
% - zuerst als einfache Tabellen
% - Als Übersicht auf die einzelnen Fächer aufgeteilt
% - Als Tortendiagramm
% - Als Balkendiagramm wobei die Abschlüsse zusammengefasst werden
%---


\section{Archäologischer Katalog der Häuser von Pompeji}
%--
% In den Anhang/Appendix soll der Katalog.
% Alle Daten sind in der data/haeuser.csv-Datei
%---



