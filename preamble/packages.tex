%!TEX root = ../document-diss.tex
\usepackage{babel} %2 Punkte = gehe eine Ebene nach oben bei den Dateien; magic comments -> nicht von lualatex erkannt, aber von Tex ; so auch nach Absturz dies als root-Dok erkannt
\usepackage{libertine}
\usepackage{xspace}

\usepackage{graphicx}
\graphicspath{{figures/}}%relativer Pafd zu den Bildern
\usepackage{multicol}%für mehrspalitigen Text

\widowpenalty=10000%zur Vermeidung einzelner Zeilen Ende Paragraph (Hurenkinder); bei zweispaltigen manchmal manuell etwas nachbessern
\clubpenalty=10000%Vermeidung einzelner Zeilen am Anfang Paragraph (Schusterjungen)


\usepackage{imakeidx}
\makeindex[
name=per,%für Personenindex
title={Index antiker Personen},
columns=3,
options={-s style/index-style.tex}
]

\usepackage{imakeidx}
\makeindex[
name=ant,%für Personenindex
title={Index antiker Werke und deren Textstellen},
columns=3,
options={-s style/index-style.tex}
]


\indexsetup{%
  level=\addsec,%entspricht section* (keine Nummer)
  firstpagestyle=scrheadings,%normale Seitenformatierung
  headers={\indexname}{\indexname},%
  noclearpage
}

\usepackage{chngcntr}%Vorraussetzung für Dußnoten durch alle Kapiteldurchnummerieren
%\counterwithout{footnote}{chapter}%Fußnoten durch alle Kapitel durchnum.
\counterwithout{table}{chapter}%Tabellen durch alle Kapitel durchnum.
\counterwithout{figure}{chapter}

\usepackage[%
	series={},
	nocritical,
	noend,
	noeledsec,
	nofamiliar,
	noledgroup%
]{reledmac}
\usepackage{reledpar}

\usepackage{tabularx}%für Tabellen
\usepackage{booktabs}%für schöne Tabellen
\usepackage{tablefootnote}

\usepackage{xcolor}%welche Farben ohne Def. zur Verfügung, s. Package-Beschr.; andere Farebn über optionale Arg. ladbar> in eckige Klammern vor xcolor

\usepackage[singlelinecheck=false]{caption}
\usepackage{chngcntr}%Anderung von Nummerierung
 \counterwithout{table}{chapter}
 
\usepackage{marginnote}

\usepackage{siunitx}%für bssere Größenangaben

\usepackage{etoolbox}%für konditioale operatoren

\usepackage{subcaption}
\usepackage[%
content={Abbildung ausgeblendet},%zum schneller kompilieren
filename,%
size=tiny,%Schriftgröße
%position=center%left, right etc.
]{draftfigure}%modifizieren von Abbildungen

\usepackage{datatool}%Auslesen von csv-Dateien
\DTLloaddb{archaeologen}{csv/archaeologen.csv}%1.: interner Name der Datenbank, 2.: Speicherort der Datei

\DTLloaddb%
  {kaiser}%internet Name
  {csv/kaiser.csv}%Speicherort Datei
  
\DTLloaddb{fachstudenten}{csv/fachstudenten.csv}

\DTLloaddb{haeuser}{csv/haeuser.csv}
  
\usepackage{databar}%für Balkendiagramme
\usepackage{datapie}%für Tortendiagramme
\usepackage{forest}%für Baumdiagramme

\usepackage[%
section=subsection,%damit kene neue Seite
nonumberlist,%keine Seitenzahl
nopostdot%kein Punkt am Ende
]{glossaries}%für Glossare
\makeglossaries%damit Namen angezeut: auf Tools, Befehle, makeglossaries, dann nochmal kompilieren


 
\setkomafont{disposition}{\normalfont}%Überschriften normal -> auch mit Serifen
%\setkomafont{chapter}{\Large\scshape}%Kaoitelüberschrift in Kapitälchen
\setkomafont{section}{\itshape}

\addtokomafont{footnotenumber}{\normalfont\sffamily}%sffamily = ohne Serifen

\deffootnote%
 {0.075cm}%Einrückung der Fußnote
 {\normalparindent}%Paragraphenabsatz
 {\makebox[\normalparindent][r]{\thefootnotemark\,}}%Leerzeichen, damit mehr Abstand nach Zahl, probieren halbes Leerzeichen : \,
 
\newlength{\normalparindent}
\AtBeginDocument{\setlength{\normalparindent}{\parindent}}

%\setfootnoterule{0pt}%wenn Trennstrich nicht gewollt
\usepackage[headsepline=.5pt]{scrlayer-scrpage}
\setkomafont{pageheadfoot}{\sffamily}

\makeatletter%nötig wegen @-Zeichen
\newcommand\BC{\,v.\,Chr\@ifnextchar.{}{.\relax\@\xspace}}%erster Fall in Klammer positiv, 2. Fall negativ, relax für Abstand nach Punkt, weil unterchiedl. nach normalen Punkt und Satzendpunkt
\newcommand\AD{\,n.\,Chr\@ifnextchar.{}{.\relax\@\xspace}}
\makeatother

\newcommand\Index[1]{\index[per]{#1}#1\xspace}
\newcommand\Indexmargin[1]{%
	\index[per]{#1}%
	\marginnote{#1}%
	#1\xspace}


\iffalse%alles nachfolgende wird ignoriert ; Strg + T setzt % vor alles markierte
\newglossaryentry{Augustus}{%
	name={\textbf{Augustus}},%
	description={Augustus (Imperator Caesar Divi Filius Augustus),%
		\newline geb. 63.\BC
		\newline gest. 14\AD
		\newline (27\BC--14\AD)},%
	first={Augustus (27\protect\BC--14\protect\AD)}%
}


\newglossaryentry{Tiberius}{%
	name={Tiberius},%
	description={Imperator 14--37\AD},%
	first={Tiberius (14\protect\AD--37\protect\AD)}%
}
\fi%und zwar bis hier

\newglossaryentry{MarcAnton}{%
	name={Marcus Antonius},
	description={Verräter},
	first={Marcus Antonius, der Verräter}%
}

\glssetexpandfield{name}
\glssetexpandfield{desc}
\glssetexpandfield{first}
\DTLforeach*{kaiser}%Datenbank
{\kaiser=Kaiser,%
\name=Name,%
\geburt=Geburt,%
\geburtsort=Geburtsort,%
\tod=Tod,%
\sterbeort=Sterbeort,%
\regierungszeit=Regierungszeit,%
\regierungszeitkurz=Regierungszeitkurz%
}
{\newglossaryentry{\kaiser}%
{name={\textbf{\kaiser}},%
description={\kaiser\ifdefempty{\name}{}{\xspace(\name)};%
\newline \geburt\xspace in \geburtsort},
first={\kaiser\ifdefempty{\regierungszeitkurz}{}{\xspace (\regierungszeitkurz)}}}}

%\renewcommand\thesection{Kat.\arabic{section}}%so Katalog Nummern

\newtoggle{hund}%der für kond. opraetor
\newtoggle{finalpdf}
\newtoggle{nocopyright}

\newcommand\missingcopyright{
\iftoggle{nocopyright}
{\setkeys{Gin}{draft}%setzt Bild auf draft
\setkeys{draftfigure}{filename=true, content={Abbildung musste aufgrund fehelender Digitalrechte ausgeblendet werden.}}}
{}}

\renewcaptionname{ngerman}{\figurename}{Abb.}